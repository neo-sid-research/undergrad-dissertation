
\chapter{Introduction}

The low signal-to-noise ratio obtained in extreme[<1 lux] low-light, single-frame, short-exposure[1/30s] images breaks most traditional denoising algorithms used in camera pipelines.
Using deep convolutional neural networks to reconstruct an image from its raw sensor data yields results that are both quantitatively and qualitatevely good.

With the See-in-the-Dark dataset published by Chen et al. in \cite{DBLP:conf/cvpr/ChenCXK18:lsid}, we will experiment adding extra supervision to achieve a better image quality.
More specifically, we will use instance segmentation labels generated by other neural networks to augment the encoded feature representation from which our image-to-image translation model decodes the enhanced images.
The intuition is that this will allow our model to more consistently color-grade the objects in even the noisier, more color-biased images.

Our experiments will consist in changing from which intermediary layer the segmentation branch of our network will fork off[picture illustrating different fork possibilities] and evaluating different strategies to automatically annotate the segmentation labels for our ground truth data.
We will evaluate performance using a perceptually-inspired image quality metric known as structural similarity index (SSIM).
The loss function used by the decoder part of the model will be the \( \ell_1 \) loss.


\section{Layout}

On chapter \ref{chap:background}, we dive into previous works and explain why we are building upon them.
On chapter \ref{chap:proposal}, we explain what we hope to accomplish and in its sections we go into detail on how we intend to do that.
On our final chapter, \ref{chap:activ_sched}, we outline our plans for the next semester by listing expected tasks on a timetable.


\chapter{\label{chap:background}Background and Related Work}

Low-light photography is difficult because noise becomes a lot more prevalent when only small amounts of signal (light) are present.
Obvious strategies to cope with low light environments are very limited, especially in extremely dark conditions such as less than 0.1 lux at the camera.
Hardware compensations could be (and they often are) manually applied by experienced photographers in controlled settings.
We will assume however that those are impractical and out of our scope, 
For example, 
High ISO only leads to higher SNRs.
physical means have their own drawbacks

\section{Denoising Algorithms}

\section{Seeing in the Dark}

\section{Multi-task learning}


\chapter{\label{chap:proposal}Proposal}

Extra supervision

\section{Methodology}

\subsection{Dataset}

\subsection{Evaluation Measures}


\chapter{\label{chap:activ_sched}Activities and Schedule}
