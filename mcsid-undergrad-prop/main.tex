
% Seleção de idioma da monografia. Por enquanto as únicas opções
% suportadas são 'portuguese' e 'english'
% Para impressão em frente e verso, use a opção 'twoside'. Da
% mesma forma, use 'oneside' para impressão em um lado apenas.
\documentclass[english, oneside]{tcc}

%----------------------------------------------------------------
% Coloque seus pacotes abaixo.
%
% Obs.: muitos pacotes de uso comum do LaTeX, como amsmath,
% geometry e url já são automaticamente incluídos pela classe
% (veja o arquivo .cls). Isso torna obrigatória a presença destes
% no sistema para o uso desta classe, mas ao mesmo tempo o uso se
% torna mais simples.  Recomendo a instalação da versão mais
% recente da distribuição TeXLive (para Windows e UNIXes):
% www.tug.org/texlive/
%
% Pacotes e opções já incluídas automaticamente:
%
% \RequirePackage[T1]{fontenc}[2005/09/27]
% \RequirePackage[utf8x]{inputenc}[2008/03/30]
% \RequirePackage[english,brazil]{babel}[2008/07/06]
% \RequirePackage[a4paper]{geometry}[2010/09/12]
% \RequirePackage{textcomp}[2005/09/27]
% \RequirePackage{lmodern}[2009/10/30]
% \RequirePackage{indentfirst}[1995/11/23]
% \RequirePackage{setspace}[2000/12/01]
% \RequirePackage{textcase}[2004/10/07]
% \RequirePackage{float}[2001/11/08]
% \RequirePackage{amsmath}[2000/07/18]
% \RequirePackage{amssymb}[2009/06/22]
% \RequirePackage{amsfonts}[2009/06/22]
% \RequirePackage{url}
% \RequirePackage[table]{xcolor}[2007/01/21]
%----------------------------------------------------------------
% Para inserção de figuras.
\usepackage{graphicx}
% Utilize a opção 'pdftex' se você estiver usando o pdflatex (que
% permite figuras em formatos como .jpg ou .png)
%\usepackage[pdftex]{graphicx}

% Para tabelas com elementos ocupando mais de uma linha
\usepackage{multirow}
% Para frações na mesma linha (ex. ⅓).
\usepackage{nicefrac}
% Para inserir figuras lado a lado.
% \usepackage{subfigure}
% Para formatar algoritmos.
% A opção [algo2e] é necessária para evitar conflitos
% com as definições da classe.
%\usepackage[algo2e]{algorithm2e}
\usepackage{algorithmic}
% Um float do tipo algoritmo. No momento
% este pacote é incompatível com a classe.
%\usepackage{algorithm}


\author{Nei Cardoso de Oliveira Neto}

\title{Podem Câmeras Móveis Ver No Escuro?}
      {Can Mobile Cameras See in the Dark?}

\tipotrabalho{\ptci}         % Proposta de Trabalho de Conclusão
%\tipotrabalho{\tci}         % Trabalho de Conclusão I
%\tipotrabalho{\tcii}        % Trabalho de Conclusão II


% Seleção do curso ("este trabalho é um requisito parcial para
% obtenção do grau de (mestre ou doutor) em Ciência da Computação").
% \curso{\cc} % Ciência da Computação
\curso{\si} % Sistemas de Informação


\orientador{Rodrigo Coelho Barros}
% \coorientador{Ciclano de Farias}

%----------------------------------------------------------------
% A capa é inserida automaticamente. Por isso não é necessário
% chamar \maketitle
%----------------------------------------------------------------
\begin{document}

%----------------------------------------------------------------
% Depois da capa vem a dedicatória e a epígrafe.
%----------------------------------------------------------------
% \dedicatoria{Dedico este trabalho a meus pais.}

% \epigrafe{The art of simplicity is a puzzle of complexity.}
%          {Douglas Horton}

%----------------------------------------------------------------
% Também dá para fazer as duas na mesma página:
%----------------------------------------------------------------
%\dedigrafe{Dedico este trabalho a meus pais.}
%          {The art of simplicity is a puzzle of complexity.}
%          {Douglas Horton}

%----------------------------------------------------------------
% A seguir, a página de agradecimentos (OPCIONAL):
%----------------------------------------------------------------
% \begin{agradecimentos}
% À lorem ipsum, dolor sit amet consetetur sadipscing elitr sed diam
% nonumy eirmod tempor. invidunt ut labore et dolore magna aliquyam

% À erad sed, diam voluptua at vero, eos et accusam et justo duo
% dolores et ea rebum stet clita.

% À kasd gubergren, no sea. takimata sanctus est lorem ipsum dolor sit
% amet lorem ipsum dolor sit amet. consetetur sadipscing elitr sed

% À diam nonumy, eirmod tempor, invidunt ut labore et dolore magna
% aliquyam erat sed diam voluptua at.
% \end{agradecimentos}

%----------------------------------------------------------------
% Resumo, com as palavras-chave passadas por parâmetro
% (OBRIGATÓRIO, ao menos para teses e dissertações)
%----------------------------------------------------------------
% \begin{resumo}{lorem, ipsum, dolor, sit, amet}
% Seu resumo em português aqui. lorem ipsum dolor sit amet
% consetetur sadipscing elitr sed diam nonumy eirmod tempor invidunt
% ut labore et dolore magna aliquyam erat sed diam voluptua at vero
% eos et accusam et justo duo dolores et ea rebum stet clita.  kasd
% gubergren no sea takimata sanctus est lorem ipsum dolor sit amet
% lorem ipsum dolor sit amet consetetur sadipscing elitr sed diam
% nonumy eirmod tempor invidunt ut labore et dolore magna aliquyam
% erat sed diam voluptua at.
% \end{resumo}


\begin{abstract}{fully convolutional network, image-to-image translation, low-light photography, perceptual quality}
Taking photos in low light environments with a mobile device such as a smartphone is challenging.
Mainly due to hardware limitations, the added noise that comes with high ISO settings, and the blur introduced by the above-average exposures necessary.
Several proprietary black-box strategies have been developed and deployed to the flagship smartphones.
In this paper, we propose an extension of the See-in-the-Dark dataset \cite{DBLP:conf/cvpr/ChenCXK18:lsid} (section 3) with images taken by smartphones, instead of DSLRs.
While we use the same U-net architecture proposed by Ronneberger et al in~\cite{DBLP:conf/miccai/RonnebergerFB15:unet} and already used on this very task by Chen et al in~\cite{DBLP:conf/cvpr/ChenCXK18:lsid} (section 4.1), we propose the use classification and segmentation in a branch of the same network image-to-image translation network to give it extra supervision in the form of context-awareness.
We also experiment with perceptual dissimilarity metrics as loss functions instead of the traditional pixel-wise MAE loss the LSID authors had originally found to be the best performing.
\end{abstract}


% \listoffigures       % Lista de figuras      (OPCIONAL)
% \listoftables        % Lista de tabelas      (OPCIONAL)
% \listofalgorithms    % Lista de algoritmos   (OPCIONAL)
% \listofacronyms      % Lista de siglas       (OPCIONAL)
% \listofabbreviations % Lista de abreviaturas (OPCIONAL)
% \listofsymbols       % Lista de símbolos     (OPCIONAL)
\tableofcontents     % Sumário               (OBRIGATÓRIO)


% \include{exemplo-cap1}
% \include{exemplo-cap2}

\chapter{Introduction}

The low signal-to-noise ratio obtained in extreme[<1 lux] low-light, single-frame, short-exposure[1/30s] images breaks most traditional denoising algorithms used in camera pipelines.
Using deep convolutional neural networks to reconstruct an image from its raw sensor data yields results that are both quantitatively and qualitatevely good.

With the See-in-the-Dark dataset published by Chen et al. in \cite{DBLP:conf/cvpr/ChenCXK18:lsid}, we will experiment adding extra supervision to achieve a better image quality.
More specifically, we will use instance segmentation labels generated by other neural networks to augment the encoded feature representation from which our image-to-image translation model decodes the enhanced images.
The intuition is that this will allow our model to more consistently color-grade the objects in even the noisier, more color-biased images.

Our experiments will consist in changing from which intermediary layer the segmentation branch of our network will fork off[picture illustrating different fork possibilities] and evaluating different strategies to automatically annotate the segmentation labels for our ground truth data.
We will evaluate performance using a perceptually-inspired image quality metric known as structural similarity index (SSIM).
The loss function used by the decoder part of the model will be the \( \ell_1 \) loss.


\section{Layout}

On chapter \ref{chap:background}, we dive into previous works and explain why we are building upon them.
On chapter \ref{chap:proposal}, we explain what we hope to accomplish and in its sections we go into detail on how we intend to do that.
On our final chapter, \ref{chap:activ_sched}, we outline our plans for the next semester by listing expected tasks on a timetable.


\chapter{\label{chap:background}Background and Related Work}

Low-light Photography text

\section{Denoising Algorithms}

\section{Seeing in the Dark}

\section{Multi-task learning}


\chapter{\label{chap:proposal}Proposal}

Extra supervision

\section{Methodology}

\subsection{Dataset}

\subsection{Evaluation Measures}


\chapter{\label{chap:activ_sched}Activities and Schedule}


\bibliographystyle{tcc-num}
\bibliography{bibliography}

%----------------------------------------------------------------
% Após \appendix, se iniciam os capítulos de Apêndice, com
% numeração alfabética.
%----------------------------------------------------------------
% \appendix
% \chapter{Meu primeiro apêndice}
% \chapter{My second appendix}

%----------------------------------------------------------------
% Aqui vão os "capítulos" de anexos. Cada anexo deve
% ser considerado um capítulo.
%----------------------------------------------------------------
% \anexos
% \chapter{Meu primeiro anexo}
% \chapter{My second attachment}

% E aqui (para a felicidade de todos) termina o documento.
\end{document}
