
% Seleção de idioma da monografia. Por enquanto as únicas opções
% suportadas são 'portuguese' e 'english'
% Para impressão em frente e verso, use a opção 'twoside'. Da
% mesma forma, use 'oneside' para impressão em um lado apenas.
\documentclass[english, oneside]{tcc}

%----------------------------------------------------------------
% Coloque seus pacotes abaixo.
%
% Obs.: muitos pacotes de uso comum do LaTeX, como amsmath,
% geometry e url já são automaticamente incluídos pela classe
% (veja o arquivo .cls). Isso torna obrigatória a presença destes
% no sistema para o uso desta classe, mas ao mesmo tempo o uso se
% torna mais simples.  Recomendo a instalação da versão mais
% recente da distribuição TeXLive (para Windows e UNIXes):
% www.tug.org/texlive/
%
% Pacotes e opções já incluídas automaticamente:
%
% \RequirePackage[T1]{fontenc}[2005/09/27]
% \RequirePackage[utf8x]{inputenc}[2008/03/30]
% \RequirePackage[english,brazil]{babel}[2008/07/06]
% \RequirePackage[a4paper]{geometry}[2010/09/12]
% \RequirePackage{textcomp}[2005/09/27]
% \RequirePackage{lmodern}[2009/10/30]
% \RequirePackage{indentfirst}[1995/11/23]
% \RequirePackage{setspace}[2000/12/01]
% \RequirePackage{textcase}[2004/10/07]
% \RequirePackage{float}[2001/11/08]
% \RequirePackage{amsmath}[2000/07/18]
% \RequirePackage{amssymb}[2009/06/22]
% \RequirePackage{amsfonts}[2009/06/22]
% \RequirePackage{url}
% \RequirePackage[table]{xcolor}[2007/01/21]
%----------------------------------------------------------------
% Para inserção de figuras.
\usepackage{graphicx}
% Utilize a opção 'pdftex' se você estiver usando o pdflatex (que
% permite figuras em formatos como .jpg ou .png)
%\usepackage[pdftex]{graphicx}

% Para tabelas com elementos ocupando mais de uma linha
\usepackage{multirow}
% Para frações na mesma linha (ex. ⅓).
\usepackage{nicefrac}
% Para inserir figuras lado a lado.
% \usepackage{subfigure}
% Para formatar algoritmos.
% A opção [algo2e] é necessária para evitar conflitos
% com as definições da classe.
%\usepackage[algo2e]{algorithm2e}
\usepackage{algorithmic}
% Um float do tipo algoritmo. No momento
% este pacote é incompatível com a classe.
%\usepackage{algorithm}


\author{Nei Cardoso de Oliveira Neto}

%----------------------------------------------------------------
% Título (OBRIGATÓRIO). Devem ser passados DOIS parâmetros,
% o título em português E o inglês, não importando o idioma
% escolhido. Os títulos são utilizados para a montagem da capa,
% resumo e abstract mais tarde.
%----------------------------------------------------------------
\title{Seu Título Aqui}
      {Can Mobile Cameras See in the Dark?}

%----------------------------------------------------------------
% Opções para o tipo de trabalho (OBRIGATÓRIO)
%----------------------------------------------------------------
\tipotrabalho{\ptci}         % Proposta de Trabalho de Conclusão
%\tipotrabalho{\tci}         % Trabalho de Conclusão I
%\tipotrabalho{\tcii}        % Trabalho de Conclusão II

%----------------------------------------------------------------
% Seleção do curso ("este trabalho é um requisito parcial para
% obtenção do grau de (mestre ou doutor) em Ciência da Computação").
%----------------------------------------------------------------
% \curso{\cc} % Ciência da Computação
\curso{\si} % Sistemas de Informação
%\curso{\es} % Engenharia de Software

%----------------------------------------------------------------
% Orientador (e Co-orientador, caso haja um). É OBRIGATÓRIO
% informar pelo menos o orientador.
%----------------------------------------------------------------
\orientador{Rodrigo Coelho Barros}
% \coorientador{Ciclano de Farias}

%----------------------------------------------------------------
% A capa é inserida automaticamente. Por isso não é necessário
% chamar \maketitle
%----------------------------------------------------------------
\begin{document}

%----------------------------------------------------------------
% Depois da capa vem a dedicatória e a epígrafe.
%----------------------------------------------------------------
% \dedicatoria{Dedico este trabalho a meus pais.}

% \epigrafe{The art of simplicity is a puzzle of complexity.}
%          {Douglas Horton}

%----------------------------------------------------------------
% Também dá para fazer as duas na mesma página:
%----------------------------------------------------------------
%\dedigrafe{Dedico este trabalho a meus pais.}
%          {The art of simplicity is a puzzle of complexity.}
%          {Douglas Horton}

%----------------------------------------------------------------
% A seguir, a página de agradecimentos (OPCIONAL):
%----------------------------------------------------------------
% \begin{agradecimentos}
% À lorem ipsum, dolor sit amet consetetur sadipscing elitr sed diam
% nonumy eirmod tempor. invidunt ut labore et dolore magna aliquyam

% À erad sed, diam voluptua at vero, eos et accusam et justo duo
% dolores et ea rebum stet clita.

% À kasd gubergren, no sea. takimata sanctus est lorem ipsum dolor sit
% amet lorem ipsum dolor sit amet. consetetur sadipscing elitr sed

% À diam nonumy, eirmod tempor, invidunt ut labore et dolore magna
% aliquyam erat sed diam voluptua at.
% \end{agradecimentos}

%----------------------------------------------------------------
% Resumo, com as palavras-chave passadas por parâmetro
% (OBRIGATÓRIO, ao menos para teses e dissertações)
%----------------------------------------------------------------
% \begin{resumo}{lorem, ipsum, dolor, sit, amet}
% Seu resumo em português aqui. lorem ipsum dolor sit amet
% consetetur sadipscing elitr sed diam nonumy eirmod tempor invidunt
% ut labore et dolore magna aliquyam erat sed diam voluptua at vero
% eos et accusam et justo duo dolores et ea rebum stet clita.  kasd
% gubergren no sea takimata sanctus est lorem ipsum dolor sit amet
% lorem ipsum dolor sit amet consetetur sadipscing elitr sed diam
% nonumy eirmod tempor invidunt ut labore et dolore magna aliquyam
% erat sed diam voluptua at.
% \end{resumo}


\begin{abstract}{fully convolutional network, image-to-image translation, low-light photography, perceptual quality}
Taking photos in low light environments with a mobile device such as a smartphone is challenging.
Mainly due to hardware limitations, the added noise that comes with high ISO settings, and the blur introduced by the above-average exposures necessary.
Several proprietary black-box strategies have been developed and deployed to the flagship smartphones.
We propose an extension of the See-in-the-Dark dataset (\cite{DBLP:conf/cvpr/ChenCXK18} section 3) with images taken by smartphones, instead of DSLRs.
While we use the same U-net architecture proposed by **\cite{} LSID section 4**, we propose the use of a perceptual dissimilarity metric as a loss function instead of the traditional pixel-wise MAE loss.
\end{abstract}


% \listoffigures       % Lista de figuras      (OPCIONAL)
% \listoftables        % Lista de tabelas      (OPCIONAL)
% \listofalgorithms    % Lista de algoritmos   (OPCIONAL)
% \listofacronyms      % Lista de siglas       (OPCIONAL)
% \listofabbreviations % Lista de abreviaturas (OPCIONAL)
% \listofsymbols       % Lista de símbolos     (OPCIONAL)
\tableofcontents     % Sumário               (OBRIGATÓRIO)


% \include{exemplo-cap1}
% \include{exemplo-cap2}
\chapter{Introduction}
Our work aims to accomplish two big objectives:
enhance the LSID pipeline so it performs better (both qualitatively and quantitatively) and make this algorithm easier to use for developers who are not deep learning researchers.

In order to achieve that, we have come up with the following goals for our project:
to expand the original SID dataset with images from mobile devices;
to experiment with loss functions based on perceptual quality metrics to try and mitigate the L1's artifacts [ref that figure where solid colors are all messy];
to experiment with extra supervision, specifically different forms of classification and segmentation to augment the image-to-image translation;
to publish high quality usable code for inferencing (i.e., running a forward pass of an image and getting its output as an image) using this pipeline. 

By publishing a version of our code licensed as free software with an easy to use API, we will enable a broader audience to experiment with this technology, benefit from it, and maybe even contribute to its development, as we have seen happen with the Tesseract open-source OCR engine [cite their github or that Google paper].

\section{Structure}
The following chapters were written considering with the final submission in mind, therefore the past tense was often used even though when referencing events that have not happened yet.
We realize this may confuse the reviewer; there is a chart on appendix ref apendicite outlining our plans to make it easier for the reader of this proposal to understand when which events will happen.
On chapter \ref{chap:background}, we dive into previous works and explain how and why we are building upon them.
On chapter ,

\chapter{\label{chap:background}Background and Related Work}
In this chapter, we will explore the academic literature, emphasizing what is relevant to our research project. 

\section{Learning to See in the Dark}
Works with RAW images...

Does the demosaicking along with the denoising...

LSID advantages over traditional pipelines...

extreme low-light imaging

\subsection{See-in-the-Dark Dataset Limitations}
The SID dataset contains \textbf{number} of pictures from that Fuji X-T2 and other number from a Sony a7s II.
Both are mirrorless cameras (footnote explaining) in the prosumer price range.
Their raw color matrices (bayer vs x-trans) are incompatibly different, making it impossible for us to train a model (that includes the demosaicking step) on, for example, the Sony train set and test it on the Fuji test set..

Chen et al argue on \cite{DBLP:conf/cvpr/ChenCXK18:lsid} that "best results will be obtained when a dedicated network is trained for a specific camera sensor".
The latest smartphones allow for very fine control over exposure and ISO, which enables us to create a dataset in the same fashion as the SID dataset, but with mobile images.
What we are thinking here is, though mobile sensors are quite different from those of DSLRs, they are considerably close to each other (oof unscientific unstained claim based on common sense).
As Chen et al also noted "experiments with cross-sensor generalization indicate that [dedicated networks] may not always be necessary [if the sensors are similar enough]".
We intend to make available a new dataset with images from mobile devices, so more research will be possible.

\section{Loss Functions for Image Restoration}
Zhao et al showed the importance of perceptually-motivated losses for image restoration tasks where humans are the ones assessing the images' quality \cite{DBLP:journals/tci/ZhaoGFK17:l1ssimloss}.
Low-light image enhancing and denoising is exactly that kind of task.

MAE loss shortcomings for this task...

SSIM advantages and disadvantages...

Where MS-SSIM comes in...

Using both L1 and MS-SSIM as shown in \cite{DBLP:journals/tci/ZhaoGFK17:l1ssimloss} allows each to compensate for the other's faults.


\chapter{Mobile Cameras: See-in-the-Dark dataset}
MCSID, pronounced "McSeed", is a new dataset we collected and made available inspired by the original SID dataset.
We photographed 600 hundred distinct settings, with as many different sensors as we can get our hands own (as long as they all output a Bayer matrix **footnote explaining**).

\section{Images}
MCSID contains several RAW short exposure images (usually very dark, noisy, and color inaccurate renditions), each paired with a reference RAW long exposure image.
All frames are aligned with each other, as the cameras used were on tripods.
In order to maximize the usefulness of our dataset, we took photos using several different exposures and, instead of a single taking singles, we took bursts.
That will enable us to better evaluate LSID's strategy against traditional burst-based denoising algorithms.
Furthermore, it will enable us and future researchers to create different approaches to the low-light imaging problem (e.g., composing N images in a single frame).

\section{Devices}
We used a Xiaomi Mi Mix S3 and a Redmii (which model?).
Here we will list their specs, which sensor they use, supported resolution, ISO, exposures, etc.

\subsection{Xiaomi Mi Mix S3}
[wip]

\subsection{Redmii}
[wip]

\section{Methods}
We took our photographs in controlled environments.
The devices were kept stable on a tripod.
We created a small Android application to take the photos automatically, so we would not need to touch our devices to set them up in between shots.


\chapter{Experiments}
[wip]

\section{Skipping Demosaicking}
[wip]

\section{Loss Functions}
[wip]

\section{Classification}
[wip]

\section{Segmentation}
[wip]


\chapter{Final Thoughts on our Contributions}
[wip]


\bibliographystyle{tcc-num}
\bibliography{bibliography}

%----------------------------------------------------------------
% Após \appendix, se iniciam os capítulos de Apêndice, com
% numeração alfabética.
%----------------------------------------------------------------
% \appendix
% \chapter{Meu primeiro apêndice}
% \chapter{My second appendix}

%----------------------------------------------------------------
% Aqui vão os "capítulos" de anexos. Cada anexo deve
% ser considerado um capítulo.
%----------------------------------------------------------------
% \anexos
% \chapter{Meu primeiro anexo}
% \chapter{My second attachment}

% E aqui (para a felicidade de todos) termina o documento.
\end{document}
